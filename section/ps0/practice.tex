\documentclass{article}
\usepackage[margin=2cm]{geometry}

\title{R practice exercises}
\date{September 1, 2015}
\author{Jarrod Millman\\ Statistics 243\\ UC Berkeley}

\begin{document}
\maketitle

Please work through this set of practice problems to make sure you have a
basic working knowledge of R.  This is not a graded assignment,
so feel free to work through these problems in the way which works best
for you.

Most of the information you will need is covered in modules 1-4 and 6 of the R
bootcamp.  If not, it will be noted (you are also expected to be able to pick
up new functions when needed).

%load, subset, index, apply, sapply, lapply, vectorized calc., writing functions

\section*{Creating datastructures}

\begin{enumerate}

\item Create the following vectors:
    \begin{enumerate}
    \item \texttt{1, 2, 3, ..., 49, 50}
    \item a logical vector that is \texttt{TRUE} exactly when the
          corresponding element of the above vector is even
    \item \texttt{50, 49, ..., 3, 2, 1}
    \item \texttt{1, 2, 3, ..., 49, 50, 49, ..., 3, 2, 1}
    \item \texttt{-10, -9, -8, ..., 8, 9, 10}
    \item \texttt{3, 6, 9, ..., 45, 48}
    \item \texttt{"3", "6", "9", ..., "45", "48"} \hspace{10mm}
          (Hint: use the previous vector) 
    \item \texttt{"a", "a", "a", "a", "b", "b", "b", "c", "c", "d"} \hspace{10mm}
          (Hint: use \texttt{rep})
    \item turn the above character vector into a factor vector 
    \item 200 numbers between -1 and 1 (inclusive) \hspace{10mm}
          (Hint: use \texttt{seq})
    \end{enumerate}

\item Create a data frame through the following steps:
    \begin{enumerate}
    \item Create a vector \texttt{1, 2, 3, ..., 49, 50} and call it \texttt{x}
    \item Create a vector by taking the cosine of x and call it \texttt{y}
    \item Create a vector by taking the tagent of y and call it \texttt{z}
    \item Create a vector by multiplying the elements of y and z and call it
          \texttt{w}
    \item Create a logical vector that is TRUE exactly when \texttt{x} is
          between 10 and 29 inclusively and call it \texttt{f}
    \item Create a data frame with column names \texttt{x, y, z, w, f} in
          that order with the obvious content and call it \texttt{df1}
    \item How would you change the names of \texttt{df1} to uppercase letters?
    \item What would you have done differently if you wanted to use \texttt{x}
          as the row names instead of making it a column?  Would you have
          needed to use \texttt{x}?
    \end{enumerate}


\end{enumerate}

\section*{Subsetting datastructures}

\begin{enumerate}

\item Create the following:
    \begin{enumerate}
    \item Create a matrix with only the numeric elements of \texttt{df1}
          and call it \texttt{m1}
    \item Create a new matrix \texttt{m2} with only the rows where
          \texttt{df1\$f} is \texttt{TRUE}
    \item Create a new data frame \texttt{df2} with only the rows where
          \texttt{z} is non-negative and has all columns but \texttt{z}
    \item Create a new data frame \texttt{df3} without the 3rd and 17th
          rows of \texttt{df1}
    \item Create a new data frame \texttt{df4} with only the even rows
          of \texttt{df1}
    \end{enumerate}


\end{enumerate}

\section*{Vectorized calculations}

\begin{enumerate}

\item Create a vector of values $e^{2x} x^{\sqrt{x}}$ for $x = 1, 1.1, 1.2, ..., 2.9, 3.0$.

\item Create the following:
    \begin{enumerate}
    \item A $5\times5$ matrix of zeros called \texttt{x}
    \item See what \texttt{row(x)} and \texttt{col(x)} return
    \item Using \texttt{row(x)} and \texttt{col(x)} create the following matrix:
      \[ \left( \begin{array}{ccccc}
      0 & 1 & 0 & 0 & 0 \\
      1 & 0 & 1 & 0 & 0 \\
      0 & 1 & 0 & 1 & 0 \\
      0 & 0 & 1 & 0 & 1 \\
      0 & 0 & 0 & 1 & 0 \end{array} \right)\] 
    \item Using \texttt{row(x)} and \texttt{col(x)} create the following matrix:
      \[ \left( \begin{array}{ccccc}
      0 & 1 & 2 & 3 & 4 \\
      1 & 0 & 1 & 2 & 3 \\
      2 & 1 & 0 & 1 & 2 \\
      3 & 2 & 1 & 0 & 1 \\
      4 & 3 & 2 & 1 & 0 \end{array} \right)\] 
    \end{enumerate}

\item Create the following matrices:
    \begin{enumerate}
    \item Using the R \texttt{outer} function (hint: look at its \texttt{FUN} argument)
      \[ \left( \begin{array}{ccccc}
      0 & 1 & 2 & 3 & 4 \\
      1 & 2 & 3 & 4 & 5 \\
      2 & 3 & 4 & 5 & 6 \\
      3 & 4 & 5 & 6 & 7 \\
      4 & 5 & 6 & 7 & 8 \end{array} \right)\]
    \item Modify what you did above
      \[ \left( \begin{array}{ccccc}
      0 & 1 & 2 & 3 & 4 \\
      1 & 2 & 3 & 4 & 0 \\
      2 & 3 & 4 & 0 & 1 \\
      3 & 4 & 0 & 1 & 2 \\
      4 & 0 & 1 & 2 & 3 \end{array} \right)\]
    \end{enumerate}

\end{enumerate}

\section*{Using \texttt{apply}, \texttt{sapply}, and \texttt{lapply}}

\begin{enumerate}

\item Normalize rows and columns
    \begin{enumerate}
    \item Create a $5\times 6$ matrix of numbers uniformly drawn from the
          interval $(1,\ 100)$ call it \texttt{m1}.
    \item Create a new matrix \texttt{m2} using \texttt{apply} to normalize
          the rows so that they sum to 1. You will need to write a little
          function to use inside of the \texttt{apply} call. Check
          the dimensions of \texttt{m2}.  Are they the same as \texttt{m1}?
          Why?
    \item Use \texttt{apply} on \texttt{m2} to verify that the rows sum to
          1.  Now use \texttt{rowSums} to do the same thing.  Why would you
          use \texttt{rowSums} instead of \texttt{apply}.
    \item Repeat the last two steps but normalize the columns.
    \end{enumerate}

\item Linear model
    \begin{enumerate}
    \item Create a vector \texttt{x} containing \texttt{1, 1.1, 1.2, ..., 9.9, 10.0}
    \item Create a vector y that is twice \texttt{x} but with standard Gaussian
          noise added
    \item Create a scatterplot
    \item Create an \texttt{lm} object where \texttt{y} depends on \texttt{x} called
          \texttt{my\_lm}
    \item Use \texttt{lapply} to find the classes of the elements of \texttt{my\_lm}
    \item Use \texttt{sapply} to find the classes of the elements of \texttt{my\_lm}
    \item Do they differ?  Why or why not?  When would they differ and when would
          they be the same?
    \end{enumerate}

\end{enumerate}


\section*{Functions}

\begin{enumerate}

\item Write a function which returns the sum of the absolute deviations from the
median of an input vector x.  Add the following:
    \begin{enumerate}
    \item Make sure the input vector x is numeric
          (hint: use R's ? function to find out how to use \texttt{stopifnot})
    \item An additional argument \texttt{na.rm} which is a logical.  If it is
          \texttt{TRUE}, the function removes all the NAs from the computation
          of the return value.  Give it a default value of \texttt{FALSE}.
    \end{enumerate}

\item Simulate a coin toss
    \begin{enumerate}
    \item Use the \texttt{sample} function to sample with replacement a vector
          of 0s and 1s with 100 elements.  Call this \texttt{x}.  Do it again
          and call it \texttt{y}.  Would it make sense to call \texttt{set.seed}
          before calling \texttt{sample}?  Why?
    \item Write a function \texttt{sum\_heads} that takes as input the number of
          desired coin flips and returns the number of heads (assume heads are
          coded by 1). Would it make sense to call \texttt{set.seed} in the body
          of your function?  Why?
    \item Create a new vector \texttt{sums} by calling \texttt{sum\_heads(200)}
          10,000 times. (Hint: use \texttt{replicate})
    \item Plot a histogram of \texttt{sums}
    \end{enumerate}

\item Write a function that takes two numeric vectors \texttt{x} and \texttt{y}
      as well as a variable \texttt{operation} with a default value of \texttt{"add"}.
    \begin{enumerate}
    \item If \texttt{operation} is \texttt{"add"}, return \texttt{x+y}
    \item If \texttt{operation} is \texttt{"subtract"}, return \texttt{x-y}
    \item If \texttt{operation} is \texttt{"multiply"}, return \texttt{x*y}
    \item If \texttt{operation} is \texttt{"divide"}, return \texttt{x/y}
    \item If \texttt{operation} isn't one of the above, return an error that
          \texttt{operation} is unknown.
    \end{enumerate}

\item Write a function that takes a vector \texttt{x} and returns a vector
      containing the cumulative sum vector.  Note that R provides a builtin
      function \texttt{cumsum} that you can use to verify that your function
      works.  You should implement this function using a \texttt{for} loop
      to make sure you understand how it works.

\end{enumerate}

\section*{Loading (and saving) data}

\begin{enumerate}

\item Load the earnings data \texttt{data/heights.dta} from the R bootcamp.
      You should already have the R bootcamp repository cloned to your
      computer (if not, you should do so now).  Don't copy the data to your
      current working directory or change your current working directory in
      R.  In other words, you will have to either specify the full path
      to the file or the relative path from your current location.  Save
      the result as \texttt{earnings}.

    \begin{enumerate}
    \item What does \texttt{class(earnings)} return?
    \item What does \texttt{str(earnings)} return?
    \item What does \texttt{length(earnings)} return?
    \item What does \texttt{dim(earnings)} return?
    \item Use \texttt{sapply} to find the class of each column of
          \texttt{earnings}.
    \item Use \texttt{sapply} to call \texttt{summary} on just the two
          height-related columns of \texttt{earnings}.
    \item Make a boxplot with just the two height-related columns of
          \texttt{earnings}. Give it a title.
    \item Make a histogram of \texttt{earnings\$yearbn}. Make sure the
          y-axis is a density not a count. Change the title and the
          label on the x-axis. Change the title and the
          label on the x-axis. 
    \end{enumerate}


\item Make sure you remember how to load CSV (e.g., \texttt{data/cpds.csv})
      and text files with white space separators (e.g., \texttt{data/stateIncome.txt}). 

\item Saving R objects
    \begin{enumerate}
    \item Use  \texttt{ls()} to examine the objects in your working environment
    \item Save some of those objects to a R data file in the directory above
          whereever your are currently located
    \item Open a new R prompt (you may want to open a new terminal and leave
          the one you are currently using alone)
    \item From your new R prompt, verify that you don't have any objects in your
          working environment
    \item Load the R data file you saved previously
    \item Use \texttt{ls()}, \texttt{class()}, \texttt{str()}, \texttt{names()},
          and anything else you can think of to exam the R objects you loaded

    \end{enumerate}

\end{enumerate}

\end{document}
